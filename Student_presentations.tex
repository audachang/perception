\documentclass[11pt, a4paper]{article}
\usepackage[utf8]{inputenc}
\usepackage[T1]{fontenc}
\usepackage{geometry}
\usepackage{titlesec}
\usepackage{enumitem}
\usepackage{xcolor}

% Page geometry
\geometry{left=2.5cm, right=2.5cm, top=2.5cm, bottom=2.5cm}

% Title formatting
\titleformat{\section}{\large\bfseries\color{blue!40!black}}{}{0em}{}[\titlerule]
\titleformat{\subsection}{\bfseries}{}{0em}{}

\begin{document}

\begin{center}
    {\LARGE \textbf{Student Presentation Schedule \& Topics}} \\
    \vspace{0.5em}
    {\large Course: Psychology of Perception}
\end{center}

\vspace{1em}

\section*{Overview}
Student groups will present on the following topics during the designated weeks. Presentations should not merely summarize the textbook but should explore real-world applications, recent research, or counter-intuitive phenomena related to the topic.

\vspace{1em}

% Week 4
\section{Week 4: The Neural Machinery of Vision (Chapters 3 \& 4)}
\textbf{Introduction:} \\
Vision begins with light striking the retina, but "seeing" occurs in the brain. This topic explores the transformation of light energy into neural signals and how the visual cortex (V1 and beyond) begins to feature-detect the world. It covers the receptive fields, the lateral geniculate nucleus (LGN), and the architectural organization of the visual cortex, laying the biological groundwork for all visual perception.

\textbf{Brainstorming Questions:}
\begin{enumerate}
    \item How does the concept of "receptive fields" explain why we see edges better than uniform surfaces?
    \item If the image on the retina is inverted and 2D, why do we perceive the world as upright and 3D?
    \item How might damage to specific areas of the visual cortex (e.g., V1 vs. V4) result in different types of blindness (e.g., blindsight or achromatopsia)?
\end{enumerate}

% Week 6
\section{Week 6: Gestalt Principles and Object Recognition (Chapter 5)}
\textbf{Introduction:} \\
The visual system does not just perceive lines and edges; it perceives whole objects. This topic delves into the Gestalt principles of organization (grouping, segregation) and modern theories of object recognition (e.g., Recognition-by-Components). It challenges students to consider how the brain separates "figure" from "ground" and solves the inverse projection problem to recognize objects from different angles.

\textbf{Brainstorming Questions:}
\begin{enumerate}
    \item Why do AI computer vision systems still struggle with tasks that are trivial for humans, such as breaking camouflage?
    \item How do "top-down" expectations influence our ability to recognize objects in a blurry or noisy image?
    \item Can we recognize a face without recognizing its individual parts? (Holistic vs. analytic processing).
\end{enumerate}

% Week 7
\section{Week 7: The Spotlight of Visual Attention (Chapter 6)}
\textbf{Introduction:} \\
We cannot process everything in our visual field simultaneously. Attention is the mechanism that selects relevant information for processing while filtering out the rest. This presentation focuses on the bottleneck of attention, exploring phenomena like inattentional blindness, change blindness, and the difference between overt (looking at) and covert (attending to) attention.

\textbf{Brainstorming Questions:}
\begin{enumerate}
    \item Is "multitasking" actually possible, or is it just rapid switching of visual attention?
    \item If we don't pay attention to an object, do we actually "see" it? (The debate on perception without attention).
    \item How do magicians use the limitations of visual attention to perform sleight-of-hand tricks?
\end{enumerate}

% Week 9
\section{Week 9: Perception for Action (Chapter 7)}
\textbf{Introduction:} \\
Perception is not just for identification; it is for interaction. This topic contrasts the "What" pathway (ventral stream) with the "Where/How" pathway (dorsal stream). It examines how we use visual information to navigate the environment, grasp objects, and coordinate movements, highlighting the concept of "affordances"—perceiving objects in terms of their potential uses.

\textbf{Brainstorming Questions:}
\begin{enumerate}
    \item Why might a patient with brain damage be unable to describe the orientation of a mail slot but still be able to post a letter through it perfectly?
    \item How do mirror neurons bridge the gap between perceiving an action and performing it?
    \item How does the brain update our perception of the world while our eyes are constantly moving (saccades)?
\end{enumerate}

% Week 10
\section{Week 10: The World in Motion (Chapter 8)}
\textbf{Introduction:} \\
Motion is a fundamental perceptual dimension that signals danger, biological life, and intent. This topic covers how the brain detects motion (Reichardt detectors), the difference between real and apparent motion (like in movies), and the corollary discharge theory, which explains why the world doesn't jump when we move our eyes.

\textbf{Brainstorming Questions:}
\begin{enumerate}
    \item Why do we sometimes perceive a stationary train as moving when the train next to us starts to leave the station?
    \item How does "biological motion" (point-light walkers) allow us to identify gender and mood from just a few moving dots?
    \item What is the evolutionary advantage of having a visual system that is hypersensitive to motion in the periphery?
\end{enumerate}

% Week 11
\section{Week 11: The Construct of Color (Chapter 9)}
\textbf{Introduction:} \\
Color is a psychophysical construct, not a physical property of objects. This presentation explores the Trichromatic Theory and Opponent-Process Theory to explain how we see millions of colors. It also addresses color constancy—the brain's amazing ability to discount the illuminant—and the philosophical implications of individual differences in color perception.

\textbf{Brainstorming Questions:}
\begin{enumerate}
    \item If a tree falls in a forest and no one is there to see it, does it have a color? (Distinguishing physics from psychophysics).
    \item How does the "The Dress" viral phenomenon illustrate the brain's assumptions about lighting conditions?
    \item How might the world look to a tetrachromat (someone with four cone types) compared to a trichromat?
\end{enumerate}

% Week 12
\section{Week 12: Depth Perception and Size Constancy (Chapter 10)}
\textbf{Introduction:} \\
The retina is a 2D surface, yet we perceive a rich 3D world. This topic investigates the monocular (pictorial) and binocular (stereoscopic) cues the brain uses to reconstruct depth. It also explores the intimate relationship between size and distance perception, often illustrated by famous illusions like the Ames Room or the Moon Illusion.

\textbf{Brainstorming Questions:}
\begin{enumerate}
    \item Why do 3D movies require glasses, and can we create 3D experiences without them (e.g., VR, holograms)?
    \item Why does the moon look so much larger on the horizon than it does high in the sky?
    \item How do artists trick the brain into seeing depth on a flat canvas?
\end{enumerate}

% Week 14
\section{Week 14: Auditory Scene Analysis (Chapter 12)}
\textbf{Introduction:} \\
In a noisy coffee shop, we can tune into a single conversation while ignoring the clatter of cups and background music. This is the "Cocktail Party Problem." This topic explores how the auditory system groups sounds by pitch, timbre, and location to segregate the auditory stream into distinct objects, similar to figure-ground segregation in vision.

\textbf{Brainstorming Questions:}
\begin{enumerate}
    \item How does the brain "fill in" missing sounds when a phoneme is covered by a cough (Phonemic Restoration Effect)?
    \item What makes a sound "musical" versus "noise" to the brain?
    \item How does hearing loss affect the cognitive load required to separate speech from background noise?
\end{enumerate}

\end{document}